\documentclass{oci}
\usepackage[utf8]{inputenc}
\usepackage{lipsum}

\title{Territorios}
\codename{territorios}

\begin{document}
\begin{problemDescription}
  Terry es fanática del Risk.
  Risk es un popular juego de estrategia, cuyo objetivo es completar una misión
  secreta dada a cada jugador conquistando los \emph{territorios} enemigos.
  No lo confundan con imitaciones, Risk es el original y único juego de
  estrategia en el que puedes conquistar el mundo.

  El tablero del juego consiste en un mapa del mundo donde los continentes están
  divididos en territorios.
  Dos territorios son adyacentes si existe una frontera entre ellos o si hay una
  línea punteada que los une lo que significa que es posible navegar entre ellos
  a través del océano.
  Durante el juego, un territorio es controlado por a lo más un jugador a la vez
  y esto se representa poniendo una o más fichas del color del jugador que lo
  controla sobre el territorio. 

  En cada turno un jugador puede decidir atacar los territorios enemigos
  para intentar conquistarlos.
  Para realizar un ataque un jugador debe escoger uno de sus territorios y
  atacar algún territorio enemigo adyacente a él.
  Si el ataque es exitoso, el jugador atacante elimina todas las fichas enemigas
  del territorio y puede colocar algunas de las suyas en este nuevo territorio
  conquistado.
  Una vez que un territorio es conquistado el jugador debe esperar al siguiente
  turno para poder atacar desde él.
  No obstante, puede seguir atacando desde los territorios que ya poseía.

  Terry está obsesionada con el Risk y a menudo pasa tardes enteras analizando
  nuevas estrategias.
  En este comento está preocupada de saber cuantos territorios podría
  conquistar como máximo luego de $K$ turnos.
  Terry comienza el juego con una cantidad $T$ de territorios.
  En el primer turno Terry puede conquistar todos los territorios adyacentes a
  alguno de estos $T$ territorios.
  En el segundo turno puede atacar desde los territorios conquistados en
  el primer turno y conquistar todos los territorios adyacentes a alguno de
  estos.
  Luego, en el tercer turno puede expandirse a partir de los territorios
  conquistados en el segundo turno y así sucesivamente hasta completar $K$
  turnos.
  Suponiendo que entremedio no pierde ningún territorio y además que en cada
  turno conquista todos los territorios posibles, a Terry le gustaría saber con
  cuantos territorios termina después del turno $K$.
\end{problemDescription}

\begin{inputDescription}
  La entrada consiste en varias líneas.

  La primera línea consiste en dos números $N$, $E$, $K$ y $T$.
  $N$ ($1 \le N \le 1\,000$) es el número de territorios en el tablero,
  $E$ ($1 \le E \le 10\,000$) es la cantidad de pares de territorios
  adyacentes,
  $K$ ($1 \le M \le 999$) es el número turnos que Terry quiere evaluar y
  $T$ es el número de territorios con el cual Terry comienza
  el juego.
  Cada territorio es identificado con un número entre 1 y $N$.

  Las siguientes $E$ líneas describen pares de territorios adyacentes.
  Cada línea contiene un par de enteros distintos $u$ y $v$ ($1\leq u,v\leq N$),
  indicando que el territorio $u$ es adyacente al $v$.

  La última línea contiene $T$ enteros, que corresponden a los territorios
  controlados por Terry al principio del juego.

  % TODO poner una foto o un dibujo vectorial de parte del tablero y explicar--
  % hacer uno de los samples basado en la figura.
\end{inputDescription}

\begin{outputDescription}
  La salida consiste en una única línea con un número indicando la cantidad
  máxima de territorios que puede controlar Terry después de $K$ turnos.
  Notar que este número incluye los territorios que poseía al inicio del juego.
\end{outputDescription}

\begin{scoreDescription}
  \score{30} Se probarán varios casos donde $T=1$, y además los
  territorios están dispuestos en una cadena, es decir, el territorio $i$
  es adyacente solo con el territorio $i-1$ y el $i+1$, salvo el primer
  territorio que solo es adyacente con el segundo y el último que solo es
  adyacente con el penúltimo (como en el primer caso de ejemplo más abajo).
  \score{30} Se probarán varios casos donde $T=1$ y sin
  restricciones adicionales.
  \score{40} Se probarán varios casos donde $T\leq N$ y sin restricciones
  adicionales.
\end{scoreDescription}

\begin{sampleDescription}
\sampleIO{sample-1}
\sampleIO{sample-2}
\sampleIO{sample-3}
\end{sampleDescription}

\end{document}
