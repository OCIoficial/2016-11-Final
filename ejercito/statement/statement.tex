\documentclass{oci}
\usepackage[utf8]{inputenc}
\usepackage{lipsum}

\title{Expansión}
\codename{expansion}

\begin{document}
\begin{problemDescription}
El juego \emph{Ataque} es una variante del juego \emph{Risk}, donde varios
jugadores se turnan para cumplir su objetivo secreto antes que los demás.
El tablero consiste en un mapa del mundo donde los continentes están divididos
en territorios. Durante el juego, cada territorio es controlado por un solo
jugador a la vez.
En su turno, cada jugador pasa por 3 fases: nuevos soldados, ataque y
reordenamiento.
Al atacar, un jugador solo puede conquistar territorios adyacentes a los
propios y por dinámica del juego no es posible atacar desde un territorio
recién conquistado hasta el turno siguiente.


\end{problemDescription}

\begin{inputDescription}
\lipsum[1]
\end{inputDescription}

\begin{outputDescription}
\lipsum[1]
\end{outputDescription}

\begin{scoreDescription}
  \score{10} Restricciones Subtarea1
  \score{10} Restricciones Subtarea2
  \score{10} Restricciones Subtarea3
\end{scoreDescription}

\begin{sampleDescription}
\sampleIO{sample-1}
\sampleIO{sample-2}
\end{sampleDescription}

\end{document}
