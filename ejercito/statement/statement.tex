\documentclass{oci}
\usepackage[utf8]{inputenc}
\usepackage{lipsum}

\title{Ejército}
\codename{ejercito}

\begin{document}
\begin{problemDescription}
  En el juego Ataque, los jugadores juegan por turnos intentando cumplir un
  objetivo secreto, único para cada jugador, antes que los demás jugadores.
  El tablero consiste en un mapa del mundo donde toda la masa terrestre del
  planeta está dividida en un número de territorios\footnote{El juego
  original tiene 42 territorios divididos en 6 continentes.}.
  Dos territorios son adyacentes si existe una frontera entre ellos o bien si
  existe una línea punteada que los conecta (esto a través de océanos).
  Durante el juego, un territorio es controlado por un solo jugador y esto se
  representa mediante una o más fichas del color del jugador que lo controla.

  Los turnos se dividen en tres etapas.
  Una de ellas consiste en la fase de ataques, durante la cual el jugador que
  juega su turno puede realizar una serie de ataques, cada uno desde alguno de
  sus territorios con 2 fichas o más a un territorio enemigo adyacente.
  Si el ataque es exitoso, el jugador elimina todas las fichas enemigas en ese
  territorio y coloca una ficha de su propio color en él.
  Como los territorios recién conquistados quedan con una ficha, no es posible
  realizar un ataque desde ellos hasta el turno siguiente (siempre que el
  jugador se encarga de poner fichas adicionales en ese territorio).

  Hobby, los fabricantes del juego, están interesados en saber cómo cambiaría
  la dinámica del juego si fuera posible atacar desde los territorios recién
  conquistados.
  Para ello, han ideado el siguiente sistema:
  \begin{enumerate}
    \item Al empezar cada turno, los territorios del jugador son marcados con
    un 0.
    \item Al atacar, cada territorio conquistado es marcado con el número del
    territorio atacante, más uno.
    Por ejemplo, si un territorio está marcado con un 4, los territorios
    conquistados desde él se deben marcar con un 5.
    \item Un territorio sólo puede atacar si tiene dos fichas o más y además
    está marcado con un número menor que $M$, donde $M$ es un límite que Hobby
    quiere definir.
    Observa que si $M$ es 1, el juego es idéntico al original.
  \end{enumerate}

  Para evaluar esta dinámica, en Hobby quieren saber cuánto se puede expandir
  un jugador en un mismo turno.
  Para ello, te piden que ignores el número de fichas y consideres solamente el
  hecho de que un territorio puede ser atacado desde un territorio propio
  marcado con $M$ o menor.
  Ten en cuenta que un territorio enemigo puede ser adyacente a más de uno de
  los territorios del jugador en turno y que esto puede afectar el número que
  será usado para marcarlo.
\end{problemDescription}

\begin{inputDescription}
  La entrada consiste en varias líneas.

  La primera línea consiste en cuatro números $N$, $E$, $M$ y $T$ separados por
  espacios.
  $N$ ($1 \le N \le 1000$) es el número de territorios del tablero,
  $E$ ($1 \le E \le ^{N(N-1)}/_2}$) es el número de pares de territorios
  adyacentes,
  $M$ ($1 \le M \le 999$) es el número que Hobby quiere introducir al juego y
  $T$ ($0 \le T \le N$) es el número de territorios del jugador que se desea
  evaluar.

  Las siguientes $E$ líneas contienen pares de números separados por espacios.
  Cada par $u$, $v$ indica que los territorios enumerados $u$ y $v$ son
  adyacentes.
  Asume que los territorios están enumerados de 1 a $N$.

  La última línea contiene $T$ enteros separados por espacios, que corresponden
  a los territorios controlados por el jugador a evaluar.

  Hay varias subtareas, descritas a continuación, que dan cuenta de distintos
  tipos de tableros:

  \paragraph{Subtarea 1 (30 puntos)} Los territorios están dispuestos en una
  cadena o línea.
  Por ejemplo, si $N$ es 5, los territorios estarán dispuestos de la forma
  A--B--C--D--E, donde `--' indica adyacencia y A, B, C, D y E son números
  entre 1 y 5.

  \paragraph{Subtarea 2 (30 puntos)} Los territorios están dispuestos como un
  árbol.
  Esto significa que entre cada par de territorios, existe un único camino que
  los conecta.
  % TODO Quizá no es necesario explicar lo que es un camino
  Un camino entre $A$ y $B$ es una secuencia de territorios adyacentes que
  comienza con $A$ y termina con $B$.

  \paragraph{Subtarea 3 (40 puntos)} Los territorios y sus adyacencias pueden
  tener cualquier forma.
  
  % TODO poner una foto o un dibujo vectorial de parte del tablero y explicar--
  % hacer uno de los samples basado en la figura.

\end{inputDescription}

\begin{outputDescription}
  La salida consiste en una única línea con el número $Z$, el máximo número de
  territorios que puede controlar el jugador a evaluar.
  Este número incluye los territorios que poseía inicialmente.
\end{outputDescription}

\begin{scoreDescription}
  \score{30} Los territorios forman una línea.
  \score{30} Los territorios forman un árbol.
  \score{40} No hay restricciones adicionales.
\end{scoreDescription}

\begin{sampleDescription}
\sampleIO{sample-1}
\sampleIO{sample-2}
\end{sampleDescription}

\end{document}
