\documentclass{oci}
\usepackage[utf8]{inputenc}
\usepackage{lipsum}

\title{Territorios}
\codename{territorios}

\begin{document}
\begin{problemDescription}
  Terry es fanática del Risk.
  Risk es un popular juego de estrategia, cuyo objetivo es completar una misión
  secreta dada a cada jugador conquistando los \emph{territorios} enemigos.
  No lo confundan con imitaciones, Risk es el original y único juego de
  estrategia que te permite conquistar el mundo.

  El tablero del juego consiste en un mapa del mundo donde los continentes están
  divididos en territorios.
  Dos territorios son adyacentes si existe una frontera entre ellos o si hay una
  línea punteada que los une lo que significa que es posible navegar entre ellos
  a través del océano.
  Durante el juego, un territorio es controlado por a lo más un jugador a la vez
  y esto se representa poniendo una o más fichas del color del jugador que lo
  controla sobre el territorio. 

  En cada turno un jugador puede decidir atacar los territorios enemigos
  para intentar conquistarlos.
  Para realizar un ataque un jugador debe escoger un territorio controlado por
  él y atacar algún territorio adyacente.
  Si el ataque es exitoso, el jugador atacante elimina todas las fichas enemigas
  del territorio y puede colocar algunas de las suyas en este nuevo territorio
  conquistado.
  Una vez que un territorio es conquistado el jugador debe esperar al siguiente
  turno para poder atacar desde él.
  No obstante, puede seguir atacando desde los territorios que ya poseía.

  Terry 

\end{problemDescription}

\begin{inputDescription}
  La entrada consiste en varias líneas.

  La primera línea consiste en cuatro números $N$, $E$, $M$ y $T$ separados por
  espacios.
  $N$ ($1 \le N \le 1000$) es el número de territorios del tablero,
  $E$ ($1 \le E \le 10^{N(N-1)/2}$) es el número de pares de territorios
  adyacentes,
  $M$ ($1 \le M \le 999$) es el número que Hobby quiere introducir al juego y
  $T$ ($0 \le T \le N$) es el número de territorios del jugador que se desea
  evaluar.

  Las siguientes $E$ líneas contienen pares de números separados por espacios.
  Cada par $u$, $v$ indica que los territorios enumerados $u$ y $v$ son
  adyacentes.
  Asume que los territorios están enumerados de 1 a $N$.

  La última línea contiene $T$ enteros separados por espacios, que corresponden
  a los territorios controlados por el jugador a evaluar.

  Hay varias subtareas, descritas a continuación, que dan cuenta de distintos
  tipos de tableros:

  \paragraph{Subtarea 1 (30 puntos)} Los territorios están dispuestos en una
  cadena o línea.
  Por ejemplo, si $N$ es 5, los territorios estarán dispuestos de la forma
  A--B--C--D--E, donde `--' indica adyacencia y A, B, C, D y E son números
  entre 1 y 5.

  \paragraph{Subtarea 2 (30 puntos)} Los territorios están dispuestos como un
  árbol.
  Esto significa que entre cada par de territorios, existe un único camino que
  los conecta.
  % TODO Quizá no es necesario explicar lo que es un camino
  Un camino entre $A$ y $B$ es una secuencia de territorios adyacentes que
  comienza con $A$ y termina con $B$.

  \paragraph{Subtarea 3 (40 puntos)} Los territorios y sus adyacencias pueden
  tener cualquier forma.
  
  % TODO poner una foto o un dibujo vectorial de parte del tablero y explicar--
  % hacer uno de los samples basado en la figura.

\end{inputDescription}

\begin{outputDescription}
  La salida consiste en una única línea con el número $Z$, el máximo número de
  territorios que puede controlar el jugador a evaluar.
  Este número incluye los territorios que poseía inicialmente.
\end{outputDescription}

\begin{scoreDescription}
  \score{30} Los territorios forman una línea.
  \score{30} Los territorios forman un árbol.
  \score{40} No hay restricciones adicionales.
\end{scoreDescription}

\begin{sampleDescription}
\sampleIO{sample-1}
\sampleIO{sample-2}
\end{sampleDescription}

\end{document}
