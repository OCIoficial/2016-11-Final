\documentclass{oci}
\usepackage[utf8]{inputenc}
\usepackage{lipsum}

\title{Bicicleta}
\codename{bicicleta}

\begin{document}

\begin{problemDescription}
Ya bien entrada la primavera, la temperatura y el aroma de los parques forman el
ambiente perfecto para los ciclistas, quienes cada tarde llenan los parques para
disfrutar de paseos en bicicleta.

Juan Paul no es uno de ellos; él detesta andar en bicicleta.
Todos los días Juan debe recorrer grandes distancias para llegar a su trabajo en
lo alto de la cordillera, donde no es posible acceder en automóvil.
% Juan debe cruzar enormes distancias en ella para llegar a su laboratorio en la
% cordillera, lejos de la ciudad, ya que no hay caminos aptos para automóviles
% que lleguen a ese lugar.
Si bien disfruta el paisaje, lo que odia es preocuparse de los \emph{cambios} de
la bicicleta.
Como el camino que debe recorrer está entremedio de la cordillera, este cambia
mucho de pendiente y debe constantemente pasar los cambios para sentirse cómodo
al pedalear.
Un cambio más alto significa más esfuerzo, pero a la vez significa mayor
velocidad.
Si la pendiente es muy grande, Juan debe pasar a cambios más bajos para no hacer
tanto esfuerzo al pedalear.
Si la pendiente disminuye Juan intentará subir el cambio para poder ir
más rápido.
% Si el cambio es muy bajo para la pendiente, es demasiado liviano y debe
% pedalear muy rápido, y si es muy alto, debe hacer mucho esfuerzo para que la
% bicicleta se mueva.

Después de estudiar detenidamente el camino, Juan ha determinado con qué cambio
se siente cómodo para distintas pendientes.
Específicamente, Juan tiene una lista de enteros en orden descendente, donde el
$i$-ésimo entero indica la pendiente a partir de la cual este cambio es
suficientemente agradable.
Por ejemplo, si la bicicleta tiene 3 cambios y la lista es $(20, 15, 10)$,
significa que Juan se siente cómodo con el primer cambio con una pendiente de
20 o menor, se siente cómodo con el segundo con una pendiente de 15 o
menor y se siente cómodo con el tercero con una pendiente de 10 o menor.

Naturalmente Juan desea ir lo más rápido posible, así que en caso de sentirse
cómodo con más de un cambio ocupará el más alto.
Por ejemplo, dada la lista anterior, con una pendiente de 16 Juan se siente cómodo
tanto con el primer cambio como con el segundo, pero preferirá el segundo por
ser el más alto.

% Para evitar lidiar con este problema, Juan ha conseguido un dispositivo que
% puede los cambios automáticamente según la velocidad de la bicicleta.
% Desafortunadamente, este sistema solo considera la velocidad de la bicicleta y
% no toma en cuenta la pendiente del terreno.
% Esto es inútil para Juan pues siempre pedalea a la misma velocidad y quiere que
% su esfuerzo se mantenga en cierto rango.
% Para ajustarlo a sus necesidades, Juan ha modificado el dispositivo para
% considerar la pendiente.
% Sin embargo, el dispositivo tiene una memoria limitada, y solo puede almacenar
% una pendiente por cambio, además de un programa computacional.
Para evitar lidiar con el problema de pasar los cambios manualmente, Juan ha
comprado una bicicleta inteligente que puede ser programada para actuar
automáticamente a las diferencias de pendiente.
La bicicleta es muy precisa y puedes ser programada para pasar instantáneamente
a cualquiera de sus cambios.

Para usar la bicicleta es necesario implementar un programa que determine cuál
es el cambio ideal dada una pendiente.
Juan Paul no es muy hábil programando, ¿podrías ayudarlo a implementar el
programa necesario para usar la bicicleta?.

% Este programa debe determinar, a partir de las pendientes asociadas a cada
% cambio, cuál es el mejor cambio a usar dada la pendiente del terreno actual.
% Juan Paul ya determinó para cada cambio cuál es la máxima pendiente tal que el
% esfuerzo que debe realizar es aceptable
% También puede escribir el programa pero tanta experimentación con su bicicleta
% lo ha atrasado con un informe de avances de investigación que debe terminar en
% 4 horas.
% ¿Puedes escribir el programa antes de que Juan termine su informe?
\end{problemDescription}

\begin{inputDescription}
La entrada consiste en tres líneas.

La primera línea contiene dos enteros $N$ y $F$, correspondientes
respectivamente al número de cambios que tiene la bicicleta y a la cantidad
de consultas que el programa debe responder.

La siguiente línea contiene $N$ enteros $g_1,g_2,\ldots, g_N$ ($-10^5\leq
g_i\leq 10^5$) donde el $i$-ésimo entero indica la pendiente a partir de la cual
este cambio es suficientemente agradable.

La última línea contiene $F$ enteros $m_j$ ($-10^5\leq m_i\leq 10^5$), donde
$m_j$ indica la pendiente del terreno en un instante dado.
Se garantiza que al menos el primer cambio es suficientemente agradable para
Juan en todo instante.
\end{inputDescription}

\begin{outputDescription}
La salida debe consistir en $F$ líneas.
La $j$-ésima línea debe contener un solo entero correspondiente al cambio ideal
para la pendiente $m_j$.
\end{outputDescription}

\begin{scoreDescription}
  \score{16} Se probarán varios casos con $N=2$ y $F\leq 1\,000$.
  \score{38} Se probarán varios casos con $2 < N \leq 1\,000$ y $F\leq 1\,000$.
  \score{46} Se probarán varios casos con $1\,000< N \leq 10\,000$ y $1\,000 <
  F\leq 10\,000$.
\end{scoreDescription}

\begin{sampleDescription}
\sampleIO{sample-1}
\sampleIO{sample-2}
\end{sampleDescription}

\end{document}
