\documentclass{oci}
\usepackage[utf8]{inputenc}
\usepackage{lipsum}

\title{Bicicleta}
\codename{bicicleta}

\begin{document}

\begin{problemDescription}
Ya bien entrada la primavera, la temperatura y el aroma de los parques es 
perfecto para los ciclistas, quienes cada tarde llenan los parques para
disfrutar de paseos en bicicleta.

Juan Paul no es uno de ellos; detesta andar en bicicleta.
Juan debe cruzar enormes distancias en ella para llegar a su laboratorio en la
cordillera, lejos de la ciudad, ya que no hay caminos aptos para automóviles
que lleguen a ese lugar.
Si bien disfruta los paisajes, lo que odia realmente es preocuparse de los
cambios, si es muy bajo, debe pedalear muy rápido, y si es muy alto, debe hacer
mucho esfuerzo para que la bicicleta se mueva.

Para evitar lidiar con este problema, Juan ha conseguido un dispositivo que
puede los cambios automáticamente según la velocidad de la bicicleta.
Desafortunadamente, este sistema solo considera la velocidad de la bicicleta y
no toma en cuenta la pendiente del terreno.
Esto es inútil para Juan pues siempre pedalea a la misma velocidad y quiere que
su esfuerzo se mantenga en cierto rango.
Para ajustarlo a sus necesidades, Juan ha modificado el dispositivo para
considerar la pendiente.
Sin embargo, el dispositivo tiene una memoria limitada, y solo puede almacenar
una pendiente por cambio, además de un programa computacional.

Este programa debe determinar, a partir de las pendientes asociadas a cada
cambio, cuál es el mejor cambio a usar dada la pendiente del terreno actual.
Juan Paul ya determinó para cada cambio cuál es la máxima pendiente tal que el
esfuerzo que debe realizar es aceptable
También puede escribir el programa pero tanta experimentación con su bicicleta
lo ha atrasado con un informe de avances de investigación que debe terminar en
4 horas.
¿Puedes escribir el programa antes de que Juan termine su informe?
\end{problemDescription}

\begin{inputDescription}
La entrada consiste en tres líneas.

La primera línea contiene dos enteros $N$ y $F$, que son respectivamente el
número de cambios que tiene la bicicleta de Juan y el número de consultas que
el programa debe responder por segundo.

La siguiente línea contiene $N$ enteros $g_i$ en orden descendente, donde el
$g$-ésimo entero indica la pendiente a partir de la cual este cambio es
suficientemente agradable para Juan.
Por ejemplo, si el quinto (5\textsuperscript{to}) número es 20 y el sexto
(6\textsuperscript{to}) es 16, entonces Juan puede usar el quinto cambio si la
pendiente es 20 o menor y el sexto si la pendiente es 16 o menor.
Naturalmente, Juan quiere que la bicicleta vaya más rápido por lo que si puede
usar dos cambios, usará el mayor (el 6\textsuperscript{to} en este caso).

La última línea contiene $F$ enteros $m_j$, donde $m_j$ indica la pendiente del
terreno en un instante dado.
Puedes asumir que al menos el primer cambio es suficientemente agradable para
Juan en todo instante.
\end{inputDescription}

\begin{outputDescription}
La salida consiste en $F$ líneas.
La $j$-ésima línea contiene un solo entero $i_j$, que corresponde al cambio ideal
para la pendiente $m_j$.
\end{outputDescription}

\begin{scoreDescription}
  \score{10} 2 cambios y 1000 queries
  \score{10} 100000 cambios y 1000 queries
  \score{10} 100000 cambios y 1000000 queries
\end{scoreDescription}

\begin{sampleDescription}
\sampleIO{sample-1}
\sampleIO{sample-2}
\end{sampleDescription}

\end{document}
