\documentclass{oci}
\usepackage[utf8]{inputenc}
\usepackage{lipsum}

\title{Bicicleta}
\codename{bicicleta}

\begin{document}

\begin{problemDescription}
Ya bien entrada la primavera, la temperatura y el aroma primaveral forman el
ambiente perfecto para los ciclistas, quienes cada tarde llenan los parques para
disfrutar de paseos en bicicleta. %Juan Paul es uno de ellos, pero no uno cualquiera.


%Juan Paul no es uno de ellos; él detesta andar en bicicleta.
%%Todos los días Juan debe recorrer grandes distancias para llegar a su trabajo en
%%lo alto de la cordillera, donde no es posible acceder en automóvil.
%% Juan debe cruzar enormes distancias en ella para llegar a su laboratorio en la
%% cordillera, lejos de la ciudad, ya que no hay caminos aptos para automóviles
%% que lleguen a ese lugar.
%Si bien disfruta el paisaje, lo que odia es preocuparse de pasar los
%\emph{cambios} de la bicicleta.
%Como el camino que debe recorrer atraviesa la cordillera, este cambia
%mucho de pendiente y Juan debe constantemente pasar los cambios para sentirse
%cómodo al pedalear.
%Un cambio más alto significa más esfuerzo, pero a la vez significa mayor
%velocidad.
%Si la pendiente es muy grande, Juan debe pasar a cambios más bajos para no hacer
%tanto esfuerzo al pedalear.
%Si la pendiente disminuye Juan intentará subir el cambio para poder ir
%más rápido.
%% Si el cambio es muy bajo para la pendiente, es demasiado liviano y debe
%% pedalear muy rápido, y si es muy alto, debe hacer mucho esfuerzo para que la
%% bicicleta se mueva.

Juan Paul gusta de subir montañas con su bicicleta.
%Todo el camino que debe recorrer Juan Paul es en subida y
El camino hacia la cima cambia continuamente
de pendiente por lo que Juan Paul debe pasar constantemente los cambios de su bicicleta para sentirse
cómodo al pedalear. 
Un cambio más bajo implica un menor esfuerzo para subir una pendiente pronunciada, 
pero también implica que va más lento.
Por el contrario, un cambio más alto le permite ir a más velocidad pero no subir
pendientes muy pronunciadas. 
%A Juan Paul le gustaría saber cuál es el mejor cambio a utilizar para cada pendiente
%en su camino.


%Si la pendiente aumenta mucho debe poner un cambio mas bajo


%Después de estudiar detenidamente el camino, Juan ha determinado a partir de que
%pendientes los cambios le parecen lo suficientemente agradables.
%Específicamente, Juan tiene una lista de enteros en orden descendente, donde el
%$i$-ésimo entero indica la pendiente a partir de la cual este cambio es
%suficientemente agradable.
%Por ejemplo, si la bicicleta tiene 3 cambios y la lista es $(20, 15, 10)$,
%significa que Juan se siente cómodo con el primer cambio con una pendiente de
%20 o menor, se siente cómodo con el segundo con una pendiente de 15 o
%menor y se siente cómodo con el tercero con una pendiente de 10 o menor.

Cada día, dependiendo de su ánimo y energía, Juan Paul construye una lista indicando la pendiente 
máxima que está dispuesto a 
%enfrentar 
subir con cada cambio. 
% La lista es una secuencia de números entre $0$ (que representa $0$ grados) 
% y $90\,000$ (que representa $90$ grados), ordenados de mayor a menor.
Esta lista es una secuencia de números ordenada de mayor a menor, donde cada
número está entre $0$ (que representa $0$ grados) y $90\,000$ (que representa
$90$ grados).
El primero de estos números indica la pendiente máxima que Juan Paul puede 
%enfrentar 
subir con el cambio $1$, el segundo la pendiente máxima que puede 
%enfrentar 
subir con el cambio $2$, y así sucesivamente.
Por ejemplo, si Juan Paul construye la lista
\[
43\, 000\;\;\;\;\; 35\, 000\;\;\;\;\; 15\, 000\;\;\;\;\; 5\, 000
\]
entonces con el cambio $1$ puede subir pendientes de $43000$ o menos, con el 
$2$ puede subir pendientes de $35000$ o menos, con el $3$ pendientes de
$15000$ o menos, y con el $4$ pendientes de $5000$ o menos.
%eso quiere decir que con el cambio $1$ puede enfrentar hasta una pendiente de $43000$, con el 
%cambio $2$ puede enfrentar hasta una pendiente de $35000$, con el cambio $3$ hasta una pendiente de
%$15000$ y con el cambio $4$ hasta una pendiente de $5000$.

Naturalmente, Juan Paul quiere ir lo más rápido posible por lo que 
siempre usará el cambio más alto que le permite subir
una pendiente en su camino.
Por ejemplo, supongamos que Juan usa la lista de arriba y debe subir un 
camino dado por la siguiente figura (en donde se han marcado
las pendientes)
\begin{center}
\input{pendiente.pspdftex}
\end{center}

En este caso, para el primer tramo de pendiente $p_1=3000$, Juan Paul puede usar el cambio $4$, 
el más rápido.
Por el contrario, para el segundo tramo $p_2=43 000$, necesita usar el cambio $1$, el más bajo. 
Y así sucesivamente, para el tramo $p_3=15 000$ debe usar el cambio $3$, para el tramo $p_4=42 000$
debe usar el cambio $1$, para $p_5=0$ debe usar el cambio $4$ y para $p_6=25 000$ debe usar
el cambio $2$.

Juan ha comprado una bicicleta inteligente que puede ser programada para 
pasar instantáneamente a cualquiera de sus cambios. 
Para usar la bicicleta Juan necesita implementar un programa que, dada
la lista que indica la pendiente máxima puede subir con cada cambio,
y dada una descripción del camino que debe subir,
el programa determine cuál es el cambio que se debe usar en cada tramo del camino.
%es el cambio ideal dada una pendiente y las preferencias de Juan.
Juan Paul no es muy hábil programando ¿Podrías ayudarlo a implementar el
programa necesario para usar la bicicleta?
 

%Naturalmente Juan desea ir lo más rápido posible, así que en caso de sentirse
%cómodo con más de un cambio ocupará siempre el más alto posible.
%Por ejemplo, dada la lista anterior, con una pendiente de 16 Juan se siente cómodo
%tanto con el primer cambio como con el segundo, pero preferirá el segundo por
%ser el más alto.

% Para evitar lidiar con este problema, Juan ha conseguido un dispositivo que
% puede los cambios automáticamente según la velocidad de la bicicleta.
% Desafortunadamente, este sistema solo considera la velocidad de la bicicleta y
% no toma en cuenta la pendiente del terreno.
% Esto es inútil para Juan pues siempre pedalea a la misma velocidad y quiere que
% su esfuerzo se mantenga en cierto rango.
% Para ajustarlo a sus necesidades, Juan ha modificado el dispositivo para
% considerar la pendiente.
% Sin embargo, el dispositivo tiene una memoria limitada, y solo puede almacenar
% una pendiente por cambio, además de un programa computacional.

%Para evitar lidiar con el problema de pasar los cambios manualmente, Juan ha
%comprado una bicicleta inteligente que puede ser programada para actuar
%automáticamente frente a las diferencias de pendiente.
%La bicicleta es muy precisa y puedes ser programada para pasar instantáneamente
%a cualquiera de sus cambios.
%
%Para usar la bicicleta es necesario implementar un programa que determine cuál
%es el cambio ideal dada una pendiente y las preferencias de Juan.
%Juan Paul no es muy hábil programando, ¿podrías ayudarlo a implementar el
%programa necesario para usar la bicicleta?.

% Este programa debe determinar, a partir de las pendientes asociadas a cada
% cambio, cuál es el mejor cambio a usar dada la pendiente del terreno actual.
% Juan Paul ya determinó para cada cambio cuál es la máxima pendiente tal que el
% esfuerzo que debe realizar es aceptable
% También puede escribir el programa pero tanta experimentación con su bicicleta
% lo ha atrasado con un informe de avances de investigación que debe terminar en
% 4 horas.
% ¿Puedes escribir el programa antes de que Juan termine su informe?
\end{problemDescription}

\begin{inputDescription}
La entrada consiste en tres líneas.

La primera línea contiene dos enteros $N$ y $T$, donde $N$ corresponde
al número de cambios que tiene la bicicleta y $T$ a la cantidad
de tramos que tiene el camino.

La siguiente línea contiene $N$ enteros $g_1,g_2,\ldots, g_N$ ordenados de mayor
a menor, todos entre $0$ y $90\,000$, donde $g_1$ indica la pendiente máxima que
Juan Paul puede subir con el cambio $1$, $g_2$ la pendiente máxima que puede
subir con el cambio $2$, etc.

La tercera línea contiene $T$ enteros $p_1,p_2,\ldots,p_T$, todos entre $0$ y $90\,000$, 
donde $p_1$ es la pendiente del tramo $1$ del camino, $p_2$ es la pendiente
del tramo $2$, etc.
Se garantiza que todas estas pendientes son menores o iguales a $g_1$ (y por lo tanto Juan Paul
puede recorrer el camino completo).
%Una pendiente negativa significa que el camino va en bajada.
%Se garantiza que al menos el primer cambio es suficientemente agradable para
%Juan en todo instante.
\end{inputDescription}

\begin{outputDescription}
La salida debe consistir en $T$ líneas.
La línea $1$ debe contener un solo entero correspondiente al cambio que se debe usar para la pendiente $p_1$,
la línea $2$ el cambio que se debe usar para la pendiente $p_2$, la línea $3$ el que se debe usar 
para la pendiente $p_3$, y así sucesivamente hasta completar las $T$ pendientes.
\end{outputDescription}

\begin{scoreDescription}
  \score{16} Se probarán varios casos con $N=2$ y $T\leq 100$.
  \score{38} Se probarán varios casos con $2 < N \leq 1\,000$ y $T\leq 100$.
  \score{46} Se probarán varios casos con $1\,000< N \leq 10\,000$ y $100 <
  T\leq 10\,000$.
\end{scoreDescription}

\begin{sampleDescription}
\sampleIO{sample-1}
\sampleIO{sample-2}
\end{sampleDescription}

\end{document}
