\documentclass{oci}
\usepackage[utf8]{inputenc}
\usepackage{lipsum}
\usepackage{xcolor}

\title{El flojo de Lucas}
\codename{lucas}

\begin{document}
\begin{problemDescription}
  Lucas siempre ha sido un niño extremadamente flojo.
  Es tan flojo que en vez de usar el tiempo haciendo sus tareas, gasta más tiempo 
  pensando cómo hacer el mínimo esfuerzo para realizarlas.
  A Lucas le encargaron escribir un documento para el colegio y 
  sentado frente al computador piensa cómo hacer para no trabajar
  tanto.

  Después de escribir algunas palabras, %no muchas, 
  Lucas se ha dado cuenta de que   %gasta demasiado 
  usa mucho tiempo en hacer correcciones, así que podría trabajar menos si
  corrigiera las palabras de manera más eficiente.
  Cuando Lucas corrige una palabra primero selecciona con el mouse una porción
  % contigua de esta 
  de la palabra y la borra. Como es flojo (creo que eso ya lo mencionamos), Lucas
  siempre elige una única porción para borrar.
%  A Lucas le da flojera cambiarse mucho entre el teclado y el mouse así que una
%  vez que borra la porción contigua inserta las letras faltantes también de
%  forma contigua desde la posición en que quedó después de borrar.
  Una vez que borra la porción seleccionada inserta las nuevas letras una a una
  desde la posición en que quedó el cursor. % después de borrar.
%  Por ejemplo, si quisiera corregir la palabra \texttt{escribiendo} y cambiarla por
%  \texttt{existiendo}, Lucas podría eliminar las letras \texttt{scrib} y luego
%  insertar \texttt{xist}.
%  A continuación se ilustra una posible forma en que Lucas puede hacer esta corrección.
  Por ejemplo, para corregir la palabra \texttt{escribiendo} y cambiarla por
  \texttt{existiendo}, podría hacerlo como se muestra a continuación.

  \newcommand{\caret}{{\color{gray}\hspace{-0.2em}\raisebox{-0.1em}{\scalebox{1.1}{|}}\hspace{-0.24em}}}
  \begin{center}
  \begin{minipage}{100pt}
    \texttt{escribiendo}

    \texttt{e\colorbox{gray}{\hspace{-0.24em}scrib\hspace{-0.24em}}iendo}

    \texttt{e\caret{}iendo}

    \texttt{ex\caret{}iendo}

    \texttt{exi\caret{}iendo}
    
    \texttt{exis\caret{}iendo}

    \texttt{exist\caret{}iendo}    
  \end{minipage}
  \end{center}

  Es decir, Lucas primero selecciona y elimina \texttt{scrib}, y luego, desde donde quedó el cursor,
  inserta las letras \texttt{x}, \texttt{i}, \texttt{s}, y \texttt{t} una a una.
  Esta no es la única forma en que Lucas podría haber corregido la palabra.
  Podría también haber borrado la porción \texttt{scribien} y luego insertado
  \texttt{xistien}. Sin embargo, en este caso debería haber insertado 7 letras, mientras
  que de la forma anterior sólo tuvo que insertar $4$.
%  Notar que después de borrar \texttt{scrib} de la primera palabra
%  todos los caracteres faltantes deben ingresarse uno tras otro entre la
%  \texttt{e} y la \texttt{i}.
  Dado que borrar no cuesta nada, todo el trabajo se gasta insertando las nuevas letras y,
  obviamente, Lucas quiere insertar la menor cantidad de letras posibles.
%  Notar que la mejor solución podría consistir en no eliminar ninguna porción de
%  la palabra original o incluso eliminarla por completo.

	Tu tarea es escribir un programa que ayude a Lucas corrigiendo palabras.
	% de la manera 	descrita arriba.
	En particular, dadas dos palabras, tu tarea es calcular la cantidad mínima de letras
	que Lucas debe insertar para convertir la primera palabra en la segunda a partir de
	borrar una única porción de la primera e insertar nuevas letras exactamente en el lugar donde las borró.
	Por ejemplo, para el caso de las palabras \texttt{escribiendo} y \texttt{existiendo}, la respuesta
	de tu programa debiera ser $4$.
	
	Nota que existen casos en donde no es necesario borrar letras de la primera palabra. 
	Por ejemplo para corregir $\texttt{tomar}$ y convertirla en $\texttt{retomar}$,
	sólo es necesario insertar \texttt{r} y \texttt{e} al principio. En este caso la respuesta sería $2$.
	Por otro lado, en un caso extremo 
	se podría tener que borrar la palabra completa para luego insertar todas las letras.
	Por ejemplo, en el caso de corregir la palabra $\texttt{gorro}$ para convertirla en $\texttt{ahorrar}$,
	dado que sólo se puede borrar una única porción de la palabra inicial,
	la única opción sería borrar la palabra $\texttt{gorro}$ completa e insertar las $7$ letras de la palabra 
	$\texttt{ahorrar}$.
	
%  Afortunadamente, el programa que está ocupando Lucas puede ser extendido
%  mediante \emph{plug-ins}, pequeños fragmentos de código que permiten a 
%  cualquier usuario añadir nuevas funcionalidades al programa.
%  Claramente Lucas es demasiado flojo para aprender a usar este sistema así que
%  necesita tu ayuda para poder agregar la funcionalidad que le servirá para
%  poder trabajar menos.

	Para la entrada de tu programa representaremos cada letra con un 
	número entero (llamado código ASCII) 
	y, por lo tanto,
%  Internamente el programa tiene asociado un entero a cada letra, el cuál es
%  llamado el código ASCII.\@
  cada palabra es representada como una secuencia de enteros.
  Por ejemplo, la palabra \texttt{escribiendo} es representada con la secuencia de enteros
  \[
  101\;\;\; 115\;\;\; 99\;\;\; 114\;\;\; 105\;\;\; 98\;\;\; 105\;\;\; 101\;\;\; 110\;\;\; 100\;\;\; 111
  \]
	mientras que la palabra \texttt{existiendo} es representada con la secuencia de enteros
	\[ 
	101\;\;\; 120\;\;\; 105\;\;\; 115\;\;\; 116\;\;\; 105\;\;\; 101\;\;\; 110\;\;\; 100\;\;\; 111%\;\;\;\phantom{99}
	\]
	Si desde la primera secuencia borramos $115$, $99$, $114$, $105$, $98$ (correspondientes a \texttt{scrib})
	e insertamos en su lugar a $120$, $105$, $115$, y $116$ 
	(correspondientes a \texttt{x}, \texttt{i}, \texttt{s}, y \texttt{t}) obtenemos la segunda secuencia.
%  Dada una secuencia inicial y una final, tu tarea es determinar cuál es la
%  mínima cantidad inserciones con que es posible transformar la primera en la
%  segunda, siguiendo las restricciones de Lucas.
%  Por ejemplo, dadas la secuencia inicial $(101,115,99,114,105,98,105,114)$ y la
%  secuencia final $(101,120,104,105,98,105,114)$, la mínima cantidad de
%  inserciones es $2$ pues se puede eliminar la porción $(115,99,114)$ de la
%  primera y luego insertar $(120,104)$ entre el 101 y el 105.
\end{problemDescription}

\begin{inputDescription}
	La entrada contiene tres líneas.
  La primera línea contiene dos enteros $N$ y $M$ ($1\leq N,M \leq 50$)
  correspondientes a los largos de la primera y la segunda palabra.
  La siguiente línea contiene una secuencia de $N$ enteros que representan a las letras de la primera palabra.
  La tercera línea contiene una secuencia de $M$ enteros que representan a las letras de la segunda palabra.
  Se garantiza que ambas secuencias serán distintas y todos los enteros de las secuencias
  serán números entre $0$ y $255$ inclusive.
\end{inputDescription}

\begin{outputDescription}
%  Debes imprimir un único entero mayor que cero correspondiente a la
%  mínima cantidad de inserciones con que es posible transformar la primera
%  secuencia en la segunda siguiendo las restricciones del problema.
%  
  Debes imprimir un único entero correspondiente a la
  mínima cantidad de letras que Lucas debe insertar para corregir la primera palabra y convertirla en la segunda 
  siguiendo la descripción del problema.
  %a partir de
	%borrar una única porción de la primera e insertar nuevas letras exactamente en el lugar donde las borró.
\end{outputDescription}

\begin{scoreDescription}
  \score{10} Se probarán varios casos donde $N > M$ y la primera palabra 
  corresponde a una extensión de la segunda palabra (como en el primer caso de los ejemplos de abajo).
  \score{15} Se probarán varios casos donde $N < M$ y la segunda secuencia 
  corresponde a una extensión de la primera (como en el segundo caso de los ejemplos de abajo).
  \score{35} Se probarán varios casos donde $N=M$ sin restricciones adicionales.
  \score{40} Se probarán varios casos sin restricciones adicionales.
\end{scoreDescription}

\begin{sampleDescription}
\sampleIO[0.6][0.3]{sample-1}
\sampleIO[0.6][0.3]{sample-2}
\sampleIO[0.6][0.3]{sample-3}
\sampleIO[0.6][0.3]{sample-4}
\end{sampleDescription}

\end{document}
