\documentclass{oci}
\usepackage[utf8]{inputenc}
\usepackage{lipsum}
\usepackage{xcolor}

\title{El flojo de Lucas}
\codename{lucas}

\begin{document}
\begin{problemDescription}
  Lucas siempre ha sido un niño extremadamente flojo.
  Es tan flojo que a menudo gasta más tiempo en pensar como trabajar menos que
  en realmente hacer sus labores.
  En este momento a Lucas le encargaron escribir un documento para el colegio.
  Sentado frente a la pantalla del computador piensa como hacer para no trabajar
  tanto.

  Después de escribir algunas palabras, no muchas, Lucas se ha dado cuenta que
  gasta demasiado tiempo en hacer correcciones, así que podría trabajar menos si
  lo hiciera de manera más eficiente.
  Cuando Lucas corrige una palabra primero selecciona con el mouse una porción
  contigua de esta y la borra.
  A Lucas le da flojera cambiarse mucho entre el teclado y el mouse así que una
  vez que borra la porción contigua inserta las letras faltantes también de
  forma contigua desde la posición en que quedó después de borrar.
  Por ejemplo, si quisiera corregir la palabra \texttt{escribir} y cambiarla por
  \texttt{exhibir}, Lucas podría eliminar las letras \texttt{scr} y luego
  insertar \texttt{xh}.
  A continuación se ilustra los pasos que lucas hace para hacer esta corrección.

  \newcommand{\caret}{{\color{gray}\hspace{-0.2em}\raisebox{-0.1em}{\scalebox{1.1}{|}}\hspace{-0.24em}}}
  \begin{center}
    \texttt{escribir}

    \texttt{e\colorbox{gray}{\hspace{-0.24em}scr\hspace{-0.24em}}ibir}

    \texttt{e\caret{}ibir}

    \texttt{ex\caret{}ibir}

    \texttt{exh\caret{}ibir}
  \end{center}

  Notar que después de borrar \texttt{scr} de la primera palabra
  todos los caracteres faltantes deben ingresarse uno tras otro entre la
  \texttt{e} y la \texttt{i}.

  A Lucas le preocupa seleccionar una porción de la palabra de forma que después
  de borrarla tenga que insertar la menor cantidad de caracteres posibles.
  Notar que la mejor solución podría consistir en no eliminar ninguna porción de
  la palabra original o incluso eliminarla por completo.

  Afortunadamente, el programa que está ocupando Lucas puede ser extendido
  mediante \emph{plug-ins}, pequeños fragmentos de código que permiten a 
  cualquier usuario añadir nuevas funcionalidades al programa.
  Claramente Lucas es demasiado flojo para aprender a usar este sistema así que
  necesita tu ayuda para poder agregar la funcionalidad que le servirá para
  poder trabajar menos.

  Internamente el programa tiene asociado un entero a cada letra, el cuál es
  llamado el código ASCII.\@
  Cada palabra es representada con la secuencia de enteros correspondientes a
  cada letra.
  Por ejemplo, la palabra \texttt{escribir} es representada con la secuencia
  $(101,115,99,114,105,98,105,114)$ mientras que la palabra \texttt{exhibir} con
  la secuencia $(101,120,104,105,98,105,114)$.
  Dada una secuencia inicial y una final, tu tarea es determinar cuál es la
  mínima cantidad inserciones con que es posible transformar la primera en la
  segunda, siguiendo las restricciones de Lucas.
  Por ejemplo, dadas la secuencia inicial $(101,115,99,114,105,98,105,114)$ y la
  secuencia final $(101,120,104,105,98,105,114)$, la mínima cantidad de
  inserciones es $2$ pues se puede eliminar la porción $(115,99,114)$ de la
  primera y luego insertar $(120,104)$ entre el 101 y el 105.
\end{problemDescription}

\begin{inputDescription}
  La primera línea de la entrada contiene dos enteros $N$ y $M$ ($1\leq N,M \leq 50$)
  correspondientes a los largos de la secuencia inicial y final respectivamente.
  La segunda línea contiene $N$ enteros entre 0 y 255 inclusive correspondientes
  a la secuencia inicial.
  La tercera línea contiene $M$ enteros también entre 0 y 255 correspondientes
  a la secuencia final.
  Se garantiza que ambas secuencias serán distintas.
\end{inputDescription}

\begin{outputDescription}
  Debes imprimir un único entero mayor o igual que cero correspondiente a la
  mínima cantidad de inserciones con que es posible transformar la primera
  secuencia en la segunda siguiendo las restricciones del problema.
\end{outputDescription}

\begin{scoreDescription}
  \score{25} Se probarán varios casos donde $N < M$ y la segunda secuencia
  corresponde a una extensión de la primera, es decir, solo es necesario
  insertar enteros al final de la primera secuencia para transformarla en la
  segunda.
  \score{35} Se probarán varios casos donde $N=M$ sin restricciones adicionales.
  \score{40} Se probarán varios casos sin restricciones adicionales.
\end{scoreDescription}

\begin{sampleDescription}
\sampleIO{sample-1}
\sampleIO{sample-2}
\sampleIO{sample-3}
\end{sampleDescription}

\end{document}
